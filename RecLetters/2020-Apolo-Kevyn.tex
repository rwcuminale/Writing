%%%%%%%%%%%%%%%%%%%%%%%%%%%%%%%%%%%%%%%%%
% Professional Formal Letter
% LaTeX Template
% Version 2.0 (12/2/17)
%
% This template originates from:
% http://www.LaTeXTemplates.com
%
% Authors:
% Brian Moses
% Vel (vel@LaTeXTemplates.com)
%
% License:
% CC BY-NC-SA 3.0 (http://creativecommons.org/licenses/by-nc-sa/3.0/)
%
%%%%%%%%%%%%%%%%%%%%%%%%%%%%%%%%%%%%%%%%%

%----------------------------------------------------------------------------------------
%	PACKAGES AND OTHER DOCUMENT CONFIGURATIONS
%----------------------------------------------------------------------------------------

\documentclass[11pt, letterpaper]{letter} % Set the font size (10pt, 11pt and 12pt) and paper size (letterpaper, a4paper, etc)

\input{structure.tex} % Include the file that specifies the document structure

%\longindentation=0pt % Un-commenting this line will push the closing "Sincerely," and date to the left of the page

%----------------------------------------------------------------------------------------
%	YOUR INFORMATION
%----------------------------------------------------------------------------------------

\Who{Richard Cuminale} % Your name

\Title{} % Your title, leave blank for no title

\authordetails{
	English Department\\ % Your department/institution
	500 Boston Post Rd.\\ % Your address
	West Haven, CT 06516\\ % Your city, zip code, country, etc
	Richard.Cuminale@nhboe.net\\ % Your email address
	Cell: (703) 785-1430\\ % Your phone number
	URL: esumsnh.net % Your URL
}

%----------------------------------------------------------------------------------------
%	HEADER CONTENTS
%----------------------------------------------------------------------------------------

\logo{logo.jpg} % Logo filename, your logo should have square dimensions (i.e. roughly the same width and height), if it does not, you will need to adjust spacing within the HEADER STRUCTURE block in structure.tex (read the comments carefully!)

\headerlineone{ENGINEERING \& SCIENCE} % Top header line, leave blank if you only want the bottom line

\headerlinetwo{UNIVERSITY MAGNET SCHOOL} % Bottom header line

%----------------------------------------------------------------------------------------

\begin{document}

%----------------------------------------------------------------------------------------
%	TO ADDRESS
%----------------------------------------------------------------------------------------

\begin{letter}{
%	Include the following if addressing the letter to a specific university
	\today\\	
%	Prof. Jones\\
%	Mathematics Search Committee\\
%	Department of Mathematics\\
%	University of California\\
%	Berkeley, California 12345
}

%----------------------------------------------------------------------------------------
%	LETTER CONTENT
%----------------------------------------------------------------------------------------

\opening{To whom it may concern:}

I would like to recommend Kevyn Apolo to your undergraduate academic program. Kevyn was my student last year in English 3, and he is my student again this year in English 4. Kevyn is a smart and capable student, but even more importantly he is one of good character.

It's hard to easily capture who Kevyn is with the right anecdote. He seemed to me at first to be a shy, quiet student, and at first I was a little worried about him. The other students in the class---and I mean everyone---would joke around with him, and it appeared to me that he was always on the receiving end. When I asked his tenth grade teacher about this at the beginning of the year, she laughed: ``You clearly haven't heard what Kevyn can lay down!'' Sure enough, as the year went on I could see that this was in fact good-natured, and pretty equal banter between him and the other students. I realized that Kevyn, who I initially worried about, was probably the most popular kid in the class.

The reason I bring this up in a recommendation letter is because I think it's a special thing both when a kid has the personality to bring the class together and when he is humble enough not to even realize how many friends he has made. It's a rare thing to bridge the gaps between cliques in schools, and Kevyn's affability does this effortlessly. Yes, he is a critical reader and lively writer, but honestly I encounter many students with strong academic English abilities. I don't encounter many who build bridges like he does.

If I remember rightly, I had a conversation with Kevyn earlier in the year where he described his ambitions to enter the health-care field in the future. If it wasn't that, I remember for sure leaving the conversation impressed with the goodness of his ambition. I believe that should you accept Kevyn to your academic community, you'll not only gain a student ready to tackle rigorous challenges ahead, but also one who will add that necessary but intangible quality that the quiet, good kids bring to the table.

\closing{Sincerely,}

%----------------------------------------------------------------------------------------
%	OPTIONAL FOOTNOTE
%----------------------------------------------------------------------------------------

% Uncomment the 4 lines below to print a footnote with custom text
%\def\thefootnote{}
%\def\footnoterule{\hrule}
%\footnotetext{\hspace*{\fill}{\footnotesize\em Footnote text}}
%\def\thefootnote{\arabic{footnote}}

%----------------------------------------------------------------------------------------

\end{letter}

\end{document}
