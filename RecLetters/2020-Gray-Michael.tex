%%%%%%%%%%%%%%%%%%%%%%%%%%%%%%%%%%%%%%%%%
% Professional Formal Letter
% LaTeX Template
% Version 2.0 (12/2/17)
%
% This template originates from:
% http://www.LaTeXTemplates.com
%
% Authors:
% Brian Moses
% Vel (vel@LaTeXTemplates.com)
%
% License:
% CC BY-NC-SA 3.0 (http://creativecommons.org/licenses/by-nc-sa/3.0/)
%
%%%%%%%%%%%%%%%%%%%%%%%%%%%%%%%%%%%%%%%%%

%----------------------------------------------------------------------------------------
%	PACKAGES AND OTHER DOCUMENT CONFIGURATIONS
%----------------------------------------------------------------------------------------

\documentclass[11pt, letterpaper]{letter} % Set the font size (10pt, 11pt and 12pt) and paper size (letterpaper, a4paper, etc)

\input{structure.tex} % Include the file that specifies the document structure

%\longindentation=0pt % Un-commenting this line will push the closing "Sincerely," and date to the left of the page

%----------------------------------------------------------------------------------------
%	YOUR INFORMATION
%----------------------------------------------------------------------------------------

\Who{Richard Cuminale} % Your name

\Title{} % Your title, leave blank for no title

\authordetails{
	English Department\\ % Your department/institution
	500 Boston Post Rd.\\ % Your address
	West Haven, CT 06516\\ % Your city, zip code, country, etc
	Richard.Cuminale@nhboe.net\\ % Your email address
	Cell: (703) 785-1430\\ % Your phone number
	URL: esumsnh.net % Your URL
}

%----------------------------------------------------------------------------------------
%	HEADER CONTENTS
%----------------------------------------------------------------------------------------

\logo{logo.jpg} % Logo filename, your logo should have square dimensions (i.e. roughly the same width and height), if it does not, you will need to adjust spacing within the HEADER STRUCTURE block in structure.tex (read the comments carefully!)

\headerlineone{ENGINEERING \& SCIENCE} % Top header line, leave blank if you only want the bottom line

\headerlinetwo{UNIVERSITY MAGNET SCHOOL} % Bottom header line

%----------------------------------------------------------------------------------------

\begin{document}

%----------------------------------------------------------------------------------------
%	TO ADDRESS
%----------------------------------------------------------------------------------------

\begin{letter}{
%	Include the following if addressing the letter to a specific university
	\today\\	
%	Prof. Jones\\
%	Mathematics Search Committee\\
%	Department of Mathematics\\
%	University of California\\
%	Berkeley, California 12345
}

%----------------------------------------------------------------------------------------
%	LETTER CONTENT
%----------------------------------------------------------------------------------------

\opening{To whom it may concern:}

I would like to strongly recommend Michael Gray to your undergraduate academic program. Michael has shown remarkable growth in the time I've known him as his English teacher, from freshmen English 1 Lab to English 3 last year. I believe Michael has the potential to be an outstanding student.

English 1 Lab is a class to help freshmen who are in danger of not being able to keep up with high school English. At the beginning of the year, Michael struggled: it was hard to find assignments in his backpack; he played games on his phone whenever he could; he was reluctant to do any work. Over the course of the year, he made progress---as most students do---and he left the class with better habits and stronger skills than he came in with. What really surprised me was encountering him again as a student in English 3 last year.

Michael came to English 3 not as a good student, but as an \emph{excellent} student. He was like a different person. Far from being reluctant to work, he never missed an assignment. Instead of playing on his phone, he raised his hand when I asked difficult questions, and he showed a strong level of intellectual self-confidence. His reading ability had grown more sophisticated and his writing compositions were thoughtful, interesting, and well-written. It was a joy to see the growth in him. And not only had he grown, but he kept the edge he had as a Freshman: Michael would question the system a bit more than other students, and think critically not only about what I instructed the kids to study, but the entire environment around him. He is an alert and active thinker.

I think Michael figured something out during Freshman and Sophomore year. I'm not sure what it was exactly, but he has always been a fan of chess (and he's quite good at it too), and I wonder if maybe he understands that actions have consequences, and you can use this principle to set yourself up for success. Michael is now a highly capable, goal-driven student who, I believe, will be a great addition to your program. I hope you give him that chance.

\closing{Sincerely,}

%----------------------------------------------------------------------------------------
%	OPTIONAL FOOTNOTE
%----------------------------------------------------------------------------------------

% Uncomment the 4 lines below to print a footnote with custom text
%\def\thefootnote{}
%\def\footnoterule{\hrule}
%\footnotetext{\hspace*{\fill}{\footnotesize\em Footnote text}}
%\def\thefootnote{\arabic{footnote}}

%----------------------------------------------------------------------------------------

\end{letter}

\end{document}
