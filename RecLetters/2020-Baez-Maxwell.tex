%%%%%%%%%%%%%%%%%%%%%%%%%%%%%%%%%%%%%%%%%
% Professional Formal Letter
% LaTeX Template
% Version 2.0 (12/2/17)
%
% This template originates from:
% http://www.LaTeXTemplates.com
%
% Authors:
% Brian Moses
% Vel (vel@LaTeXTemplates.com)
%
% License:
% CC BY-NC-SA 3.0 (http://creativecommons.org/licenses/by-nc-sa/3.0/)
%
%%%%%%%%%%%%%%%%%%%%%%%%%%%%%%%%%%%%%%%%%

%----------------------------------------------------------------------------------------
%	PACKAGES AND OTHER DOCUMENT CONFIGURATIONS
%----------------------------------------------------------------------------------------

\documentclass[11pt, letterpaper]{letter} % Set the font size (10pt, 11pt and 12pt) and paper size (letterpaper, a4paper, etc)

\input{structure.tex} % Include the file that specifies the document structure

%\longindentation=0pt % Un-commenting this line will push the closing "Sincerely," and date to the left of the page

%----------------------------------------------------------------------------------------
%	YOUR INFORMATION
%----------------------------------------------------------------------------------------

\Who{Richard Cuminale} % Your name

\Title{} % Your title, leave blank for no title

\authordetails{
	English Department\\ % Your department/institution
	500 Boston Post Rd.\\ % Your address
	West Haven, CT 06516\\ % Your city, zip code, country, etc
	Richard.Cuminale@nhboe.net\\ % Your email address
	Cell: (703) 785-1430\\ % Your phone number
	URL: esumsnh.net % Your URL
}

%----------------------------------------------------------------------------------------
%	HEADER CONTENTS
%----------------------------------------------------------------------------------------

\logo{logo.jpg} % Logo filename, your logo should have square dimensions (i.e. roughly the same width and height), if it does not, you will need to adjust spacing within the HEADER STRUCTURE block in structure.tex (read the comments carefully!)

\headerlineone{ENGINEERING \& SCIENCE} % Top header line, leave blank if you only want the bottom line

\headerlinetwo{UNIVERSITY MAGNET SCHOOL} % Bottom header line

%----------------------------------------------------------------------------------------

\begin{document}

%----------------------------------------------------------------------------------------
%	TO ADDRESS
%----------------------------------------------------------------------------------------

\begin{letter}{
%	Include the following if addressing the letter to a specific university
	\today\\	
%	Prof. Jones\\
%	Mathematics Search Committee\\
%	Department of Mathematics\\
%	University of California\\
%	Berkeley, California 12345
}

%----------------------------------------------------------------------------------------
%	LETTER CONTENT
%----------------------------------------------------------------------------------------

\opening{To whom it may concern:}

I would like to recommend strongly Maxwell Baez to your undergraduate academic program. 
I know Max from teaching him both last year when he was a Junior in English 3 and this year in English 4. 
I can say with confidence that Max will be a strong undergraduate student because he has consistently shown himself to be an excellent, authentically engaged and positive student in both of these classes.

The first thing that comes to mind as I reflect on the experience of teaching Max is the word "bright." 
Among the students, Max is always alert, focused, and picking up on the themes, issues, and bigger questions in my instruction. 
During discussions he is outgoing and intelligent, and he brings this activity of mind to his writing too. 
Max is a student rewarding to teach, because he gives the sense---both through his work and his attitude---that he is making the most of the instruction he receives. 
It is no surprise then to hear that he is also highly responsible, never missing an assignment, and always keeping communication open. 
A few weeks ago, for instance, Max let me know he was having his wisdom teeth pulled, and he requested an extension on a paper for a couple of days. 
I have to insist on a longer extension, because I know Max is the kind of person who would have pushed himself during his recovery to make that paper happen.

Not only is Max hard-working, but he is cheerful too. 
This is not a "driven" student as much as a mature student, I think. 
He seems to have balance, and his high marks seem to come from hard work unburdened by anxiety. 
He can be put in any group and he has a friend there, and is prepared to take a leadership role. 
He has the respect of his peers and my own colleagues. 
I can't think of a better combination than a skilled and affable student, and it's my impression that this grounded attitude is a significant factor in his current success.

In the future, I see Max making himself quickly a valuable member of your academic community. 
Your professors will enjoy teaching him, and he will make the most of the opportunities you provide him. 
It's my hope that you give him this chance.

\closing{Sincerely,}

%----------------------------------------------------------------------------------------
%	OPTIONAL FOOTNOTE
%----------------------------------------------------------------------------------------

% Uncomment the 4 lines below to print a footnote with custom text
%\def\thefootnote{}
%\def\footnoterule{\hrule}
%\footnotetext{\hspace*{\fill}{\footnotesize\em Footnote text}}
%\def\thefootnote{\arabic{footnote}}

%----------------------------------------------------------------------------------------

\end{letter}

\end{document}
